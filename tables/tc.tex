\begin{table}
\renewcommand{\arraystretch}{1.5}
  \centering
  % \resizebox{\columnwidth}{!}{

    \begin{tabular}{ | c | p{5.9cm} | }
      \hline
      Function          & Description\\
    \hline
    \hline
    \funcjoin{}         & The \acs{TC} authenticates to the Issuer.
    If the \ac{TC} is genuine and admissible, the Issuer generates a long-term
    credential which is then stored in \acs{TC} memory.\\
    \hline
    \funcissue{}        & A new pseudonym is derived from the long-term
    credential and stored in \acs{TC} memory.\\
    \hline
    \funcsign{}         & Upon request from the \acs{OBU}, a \acs{V2V} message
    is signed using one of the \acs{TC} pseudonyms, along with additional
    metadata such as a timestamp.\\
    \hline
    \funcverify{}       & Given a \acs{V2V} message as input, the \acs{TC}
    checks its authenticity and freshness, returning either true (message valid)
    or false (message not valid).\\
    \hline
    \funcheartbeat{}    & Given a \acs{HB} as input, the \acs{TC} checks its
    authenticity and freshness, then inspects the \acs{PRL} and revokes any of
    its credentials that are on the list.\\
    \hline
  \end{tabular}
  % }
  \vspace{0.2cm}
  \caption{ High-level description of the main \acs{TC} functions used in our
  design. The \funcheartbeat{} function replaces \funcrevokedaa{} in earlier
  \ac{DAA} schemes. }
  \label{tbl:tc-functions}
  %\vspace{-5mm}
\end{table}
%


