\section{Summary \& Conclusions}
%
Revocation of credentials and pseudonymous identities is a critical aspect
of security and trust management in \ac{V2X} systems, necessary to prevent
malicious actors from manipulating \ac{V2X} communication for personal gain
or to harm people and infrastructure. In this paper, we
present and evaluate a self-revocation scheme that addresses scalability and
security issues of existing approaches, and we formally prove that
revocation in our scheme cannot be bypassed and completes with a
configurable upper time bound.
%
Furthermore, our approach is compatible with recent standardization efforts in
the ETSI framework in Europe and the security credential management system in
the US, and also with privacy-preserving pseudonym schemes that rely on Direct
Anonymous Attestation.
%
Compared to other solutions, our design requires a very limited communication
bandwidth and we argue that the distributed nature of our scheme reduces
computational resource utilization, specifically at the infrastructural layer.
We show that our approach scales well to large numbers of revocations and allows
for revocation times in the order of seconds, outperforming even short-lived
pseudonyms. We conclude that approaches that leverage trusted computing
technologies, such as ours, come with improved scalability and performance
characteristics that warrant inclusion in future standardization efforts in
application scenarios that deal with large numbers of participants and critical
low-latency communications, beyond \ac{V2X} and \ac{ITS}. Future work needs to
investigate privacy-preserving solutions to block malicious actors from
re-joining \ac{ITS} networks after revocation, without incurring availability
issues for well-behaving actors when losing connectivity.

