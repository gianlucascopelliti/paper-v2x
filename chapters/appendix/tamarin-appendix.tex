\section{Tamarin Models}
\label{appendix:tamarin}

This appendix extends \cref{section:design-verification}, providing more details
on how we translated our design to \tamarin{} and how we captured the properties
of our design. Here, we also discuss our variant where a trusted time source in
the \ac{TC} is available (cf.~\cref{section:design-extensions}).

\subsection{Mapping our Design to \tamarin{}}

To map our design to Tamarin, we used only six main rules:
%
\begin{itemize}
    \item \tamrule{RA\_generate\_heartbeat}: This rule takes as input the
    current time and \ac{PRL} and produces a new \ac{HB} as output, signed
    with the \ac{RA}'s private key. The \ac{HB} is also sent out to the network
    channel, and an action fact \tamrule{HeartbeatGenerated(HB)} is produced.
    This rule can only be executed once per \ac{PRL} and time step to reduce
    the number of states, since multiple execution with the same parameters
    would generate the same \ac{HB}, which is not relevant in our model since
    \tamarin{} already allows replays of network messages.
    \item \tamrule{RA\_issue\_revocation}: Taking as input the current
    time, \ac{PRL}, and a pseudonym's public key, this rule generates a
    new \ac{PRL} by adding the pseudonym to the list and incrementing the
    sequence number by one. We added a restriction to revoke each pseudonym only
    once. After execution, the action fact \tamrule{RevocationIssued(ps, t)} is
    generated.
    \item \tamrule{TC\_get\_pseudonym}: This rule allows the \ac{TC} to obtain a
    new pseudonym credential. How exactly this credential is obtained (e.g., via
    a communication to the infrastructure) is outside of our scope; What is
    important in our model is that the \ac{TC} can obtain and use an arbitrary
    number of pseudonyms. The only restriction of this rule is to ensure that
    only non-revoked \acp{TC} can obtain new pseudonyms
    (cf.~\cref{req:pseudonym}).
    \item \tamrule{TC\_process\_heartbeat}: The \ac{TC} processes a \ac{HB}
    taken as input from the channel. Restrictions ensure that the received
    \ac{HB} has a valid signature and timestamp. In addition, the \ac{TC} has to
    check whether to perform self-revocation or not. To do so, we divided this
    rule in two mutually-exclusive rules that produce different results: if any
    of the \ac{TC} pseudonyms is in the \ac{PRL}, the action fact
    \tamrule{Revoked(t)} is produced, otherwise it is not. Either way, a
    \tamrule{HeartbeatProcessed(HB,t)} action fact are produced, along with a
    new \tamrule{!Time(t\_hb)} persistent fact to synchronize the \ac{TC} time.
    \item \tamrule{TC\_sign\_message}: The \ac{TC} generates a new \ac{V2V}
    message signed with a pseudonym's private key. It takes the latest time as
    input, while restrictions ensure that the TC has not been previously revoked
    (i.e., the \tamrule{Revoked(t)} action fact does not exist). The rule
    produces a signed timestamped message and sends it to the out channel. A
    \tamrule{Signed(msg,ps)} action fact is produced as well.
    \item \tamrule{process\_message}: This rule models a generic third entity
    that receives a \ac{V2V} message from network and verifies it. If the
    signature is valid and the message is fresh, a
    \tamrule{MessageAccepted(msg,ps,t)} action fact is generated.
\end{itemize}

\subsection{Trusted Time Source in \acp{TC}}

\begin{figure}
  \centering
    \lstinputlisting[language=Tamarin]{properties-time.spthy}
  \caption{Properties of our design with a trusted time source in \ac{TC}
  translated to \tamarin{} lemmas.}
  \label{listing:tamarin-lemmas-time}
\end{figure}

The design that models a trusted time source in \ac{TC} comes with a major
difference, i.e., that \ac{TC} and \ac{RA} use the same \tamrule{!Time} facts.
That is, this means that their clock is perfectly synchronized, and the \ac{TC}
does not need to process \acp{HB} to advance its internal time.

Table~\ref{tbl:tamarin-lemmas-trusted-time} provides the full list of lemmas
used in this variant, most of which are analogous to the main model. The main
difference is that we added three \emph{reusable lemmas}, i.e., lemmas that
\tamarin{} can leverage to prove other lemmas. This was helpful since we
experienced that the two proof lemmas were harder to verify efficiently than in
our main model. By defining additional lemmas marked as \tamrule{reuse}, we
could instead build efficient proofs, allowing \tamarin{} to terminate in only a
few minutes.


\cref{listing:tamarin-lemmas-time} shows the proof properties of this model. Two
are the main differences compared to the main model: First, the
\tamrule{effective\_revocation} lemma can now directly prove our first property
to calculate the exact value for \paramteff. Indeed, we now assume that all
\acp{TC} are perfectly synchronized with the \ac{RA} time, hence the revocation
time is now absolute and comes at $\paramtrev + 2\paramtt$. Second, thanks to
internal timeouts in \ac{TC}, we can have a stronger proof for the second
property, saying that not only does the \ac{TC} not process \acp{HB} generated
after $\paramtrev{} + \paramtt{}$, but also that the \ac{TC} cannot perform
\emph{any} operations after that time because the automatic revocation timeout
would be triggered. To prove this, we defined an additional rule called
\tamrule{TC\_do\_operation}, which models a generic operation performed by the
\ac{TC} via the \tamrule{NewOperation(t)} action fact. This is then used in
\tamrule{no\_operations\_after\_timeout} to prove that there cannot be
\tamrule{NewOperation(t)} action facts with a value for $t$ bigger than
$\paramtrev + \paramtt$.

\begin{table*}
    \renewcommand{\arraystretch}{1.2}
    \newcolumntype{L}{>{\raggedright\arraybackslash}X}
      \centering
      \begin{tabularx}{.97\linewidth}{ | l | c | c | c | L | }
        \hline
        Lemma                                                     & Type   &  Oracle      & Steps & Description  \\
        \hline
        \hline
        \tamrule{sign\_possible}                                  & S      &              & 6     & Ensures that the \ac{TC} can sign \ac{V2V} messages \\
        \hline
        \tamrule{generate\_hb\_possible}                          & S      &              & 5     & Ensures that the \ac{RA} can generate and distribute \acp{HB} \\
        \hline
        \tamrule{issue\_revocation\_possible}                     & S      &              & 5     & Ensures that the \ac{RA} can revoke pseudonyms, adding them to the \ac{PRL} \\
        \hline
        \tamrule{processing\_hb\_possible}                        & S      &              & 9     & Ensures that the \ac{TC} can receive and process a \ac{HB} \\
        \hline
        \tamrule{revocation\_possible}                            & S      &              & 13    & Ensures that the \ac{TC} can self-revoke when receiving a \ac{HB} \\
        \hline
        \tamrule{process\_message\_possible}                      & S      &              & 7     & Ensures that a third entity can process \ac{V2V} messages signed by the \ac{TC}  \\
        \hline
        \tamrule{exists\_par\_tv\_2}                              & S      &              & 5     & Used to verify that \paramtt{} is arbitrary \\
        \hline
        \tamrule{exists\_par\_tv\_4}                              & S      &              & 7     & Used to verify that \paramtt{} is arbitrary \\
        \hline
        \tamrule{no\_signing\_after\_timeout}                     & F      &              & 205   & Ensures that the \ac{TC} cannot make signatures after performing automatic revocation \\
        \hline
        \tamrule{no\_signing\_after\_revocation}                  & F      &              & 2     & Ensures that the \ac{TC} cannot make signatures after performing self-revocation \\
        \hline
        \tamrule{all\_heartbeats\_processed\_within\_tolerance}   & F      & \faCheck     & 260   & Ensures that outdated \ac{HB} are correctly discarded by the \ac{TC} \\
        \hline
        \tamrule{all\_messages\_accepted\_signed\_exists}         & R      & \faCheck     & 62    & Ensures that, for each \ac{V2V} message verified by a third entity, the same message was previously signed by the \ac{TC} \\
        \hline
        \tamrule{all\_messages\_accepted\_within\_tolerance}      & R      & \faCheck     & 282   & Ensures that outdated \ac{V2V} messages are correctly discarded by the third entity \\
        \hline
        \tamrule{no\_messages\_accepted\_after\_revocation}       & R      & \faCheck     & 74    & Ensures that, if revocation is mandated at time \paramtrev, all TC messages accepted by a third entity cannot contain timestamp greater than $\paramtrev{} + \paramtt{}$ \\
        \hline
        \tamrule{effective\_revocation}                           & P      & \faCheck     & 4     & Verifies property \ref{property:revoke} in \cref{section:design-properties} \\
        \hline
        \tamrule{no\_operations\_after\_timeout}                  & P      & \faCheck     & 76    & Verifies property \ref{property:prl} in \cref{section:design-properties} \\
        \hline
      \end{tabularx}
      \vspace{0.2cm}
      \caption{Full list of lemmas proven with \tamarin{} on our model with
      trusted time. \emph{Type} is either \emph{S} (sanity), \emph{F}
      (functional), \emph{R} (reuse) or \emph{P} (proof). \emph{Oracle}
      indicates whether an oracle was needed, while \emph{Steps} the number of
      steps required by \tamarin{} to prove the lemma.}
      \label{tbl:tamarin-lemmas-trusted-time}
      %\vspace{-5mm}
    \end{table*}
    %


