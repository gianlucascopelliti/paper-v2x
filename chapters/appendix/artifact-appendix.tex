\section{Artifact Appendix}

This Appendix contains a complete description on the artifacts presented in our
paper, along with detailed instructions for running the artifacts locally and
reproducing our results.

\subsection{Description \& Requirements}
\label{artifact:requirements}

\begin{table}
\renewcommand{\arraystretch}{1.2}
  \centering
  \begin{tabular}{ | c | c | }
    \hline
    Software                    & Tested version                            \\
    \hline
    \hline
    Docker                      & 23.0.2                                    \\
    \hline
    Docker Compose              & 2.17.2                                    \\
    \hline
    kubectl                     & 1.28.2                                    \\
    \hline
    Minikube                    & 1.31.2 (running Kubernetes 1.27.4)        \\
    \hline
  \end{tabular}
  \vspace{0.2cm}
  \caption{List of all software dependencies required to run our artifacts, along with the version that we used in our setup.}
  \label{tbl:artifact-sw}
  %\vspace{-5mm}
\end{table}
%




This section provides all the information necessary to recreate the experimental
setup to run our artifacts. All experiments can run on a commodity desktop
machine. For experiment (E2), we provide a scaled-down configuration that can
still provide meaningful results. See \cref{artifact:evaluation} for more
details.

\subsubsection{How to access}
The artifacts are publicly available on Github\footnote{\repo{}}. While the
\texttt{main} branch contains the latest version of the code, the
\texttt{ndss-24-artifacts} tag contains the exact version of the code that was
submitted for review in the artifact evaluation. The latter is also available on
Zenodo~\cite{supplMaterial}.

\subsubsection{Hardware dependencies}
Our artifacts can be run on a commodity desktop machine with a x86-64 CPU. To
ensure that all artifacts run correctly, it is recommended to use a machine with
at least 8 cores and 16 GB of RAM.

\subsubsection{Software dependencies} 
A recent Linux operating system is required, preferably one between Ubuntu
$\geq$ 20.04 and Debian $\geq$ 10. Our artifacts have been tested on Ubuntu
22.04 LTS (Jammy). Additional dependencies are listed in \cref{tbl:artifact-sw}.

\subsubsection{Benchmarks} 
None.

\subsection{Artifact Installation \& Configuration}
\label{artifact:configuration}

We first require that our repository is downloaded to a local folder, e.g., via
\texttt{git clone}. To install all required dependencies, we then provide an
\texttt{install.sh} script that can be used by Ubuntu and Debian users to
install up-to-date versions of the packages listed in \cref{tbl:artifact-sw}.
Alternatively, such dependencies can be installed manually via official
channels. Note that we require running the installation script and subsequent
experiments using a non-root user with \texttt{sudo} privileges.

All artifacts leverage Docker containers for ease of use. However, except for
experiment (E2), all experiments can be run locally. Additionally, most
experiments have customizable parameters. In this Appendix, we provide
instructions for running all experiments with Docker using a fixed set of
parameters, but our repository contains extensive instructions for customizing
each experiment.

\subsection{Experiment Workflow}

Our artifacts contain three independent experiments. The first consists in the
formal verification of our design using the Tamarin prover. The second
experiment simulates a \ac{V2X} scenario on Kubernetes, where our scheme is
evaluated under different conditions. The third experiment performs a
statistical evaluation on the size of \ac{PRL} to assess the scalability of our
approach.

The proposed workflow runs the three experiments sequentially. Our repository
contains scripts and a Makefile that can be used to automate all experiments.
For brevity, this Appendix only illustrates the \texttt{make} commands to run at
each step, while extensive documentation is provided in our repository.

\subsection{Major Claims}

\begin{itemize}
    \item (C1): Our scheme guarantees revocation within a deterministic upper
    bound \paramteff, even in case of delayed or dropped revocation requests.
    This is proven by experiment (E1), whose results are reported in
    \cref{section:design-verification}, Appendix~\ref{appendix:tamarin} and
    \cref{tbl:tamarin-lemmas-main,tbl:tamarin-lemmas-trusted-time}.
    \item (C2): Our scheme is resilient to severe network malfunctions and
    delays that may occur in a real-world scenario. Even in these harsh
    conditions, revocation times are not only within the upper bound proven in
    C1, but also significantly shorter on average. This is proven by experiment
    (E2), whose results are reported in \cref{section:eval-sim} and
    \cref{fig:eval-sim-base}. Additional experiments are described in our Github
    repository.
    \item (C3): Our scheme scales well with the number of vehicles and attackers
    in the network. Additionally, even when choosing a small \paramteff{}, the
    required network and computational resources are still low. This is proven
    by experiment (E3), whose results are reported in
    \cref{section:eval-prl-size}, Appendix~\ref{appendix:markov} and
    \cref{fig:eval-prl-size,fig:eval-tv,fig:appendix:markov-e,fig:appendix:markov-n}.
\end{itemize}

\subsection{Evaluation}
\label{artifact:evaluation}

This section includes all the operational steps and experiments which must be
performed to evaluate our artifacts and validate our results. In total, all
experiments require between 30 and 60 human-minutes and around 8 compute-hours.
We assume that the machine is configured correctly with required dependencies,
as described in \cref{artifact:requirements,artifact:configuration}. The same
instructions, along with additional details, are provided in the top-level
README file of our repository.

\subsubsection{Experiment (E1) - Claim (C1)}
\label{artifact:proofs}
[Tamarin proofs] [2 human-minutes + 10 compute-minutes]: The experiment consists
in using the open-source Tamarin prover tool (version 1.6.1\footnote{The
recently-released version of Tamarin 1.8.0 is not able to verify our models
efficiently. We are currently investigating this issue.}) to verify our
revocation design. Tamarin receives as input our model specification along with
the properties (\textit{lemmas}) we want to verify, and attempts to verify such
properties. Our artifacts include two separate models, one that illustrates the
main design in \cref{chapter:design} and another that illustrates a variation of
our design where \acs{TC} have a trusted time that is synchronized with the
\acs{RA} (\cref{section:design-extensions}). See
\cref{section:design-verification} and Appendix~\ref{appendix:tamarin} for more
information.

\textit{[Preparation]} In a new shell, go to the \texttt{proofs} folder.

\textit{[Execution]} Firstly, prove the main design, which is called
\texttt{centralized-time}. The command to run, along with the expected output,
is provided below.

\begin{lstlisting}[language=bash]
# Prove the `centralized-time` model (5 minutes)
# Expected output: 
# - Tamarin prints a "summary of summaries" at the end
# - In the summary, all lemmas are marked as "verified"
# - a `output_centralized.spthy` file under `./out`
make prove MODEL=centralized-time OUT_FILE=output_centralized.spthy
\end{lstlisting}

Secondly, prove the variation of our design with trusted time, which is called
\texttt{distributed-time}. The command to run, along with the expected output,
is provided below.

\begin{lstlisting}[language=bash]
# Prove the `distributed-time` model (5 minutes)
# Expected output: 
# - Tamarin prints a "summary of summaries" at the end
# - In the summary, all lemmas are marked as "verified"
# - a `output_distributed.spthy` file under `./out`
make prove MODEL=distributed-time OUT_FILE=output_distributed.spthy
\end{lstlisting}
    
\textit{[Results]} Upon completion, Tamarin prints to standard output a summary
where each lemma is marked either as \emph{verified}, \emph{falsified}, or
\emph{analysis incomplete}. In our case, the experiment can be considered
successful if all lemmas are marked as \emph{verified}. The output files
\emph{output\_\{centralized,distributed\}.spthy}, contain the full proofs
computed by Tamarin.

\subsubsection{Experiment (E2) - Claim (C2)}
\label{artifact:simulations}
[Simulations] [20-30 human-minutes + 4.5-5 compute-hours]: The experiment
consists in simulating a \acs{V2X} edge area with a number of vehicles that
interact with each other, some of them being malicious. The goal of this
simulation is collecting data on revocation times for malicious vehicles, and
analyzing their distribution compared to different scenarios and attacker
levels. See \cref{section:eval-sim} for more information. The simulation runs on
a Kubernetes cluster: Our experiments have been carried out on a large cluster
with 8 nodes, spawning 400 vehicles and with a total simulation time of around
32 hours. For the artifact evaluation, however, we propose a \emph{scaled-down}
configuration that can run locally on a Minikube instance and spawns 50
vehicles, with a total simulation time of around 4 hours. We believe that this
configuration is still acceptable since it only affects the quantity of gathered
data (i.e., number of revocations and revocation times). The output plots will
therefore be comparable to the figures in our paper, although less data will
produce less outliers. Besides, given that there is a certain degree of
randomicity (due to simulated network conditions), it is impossible to reproduce
\emph{exactly} the same results that are shown in the paper: Each simulation,
therefore, will produce slightly different box plots.

\textit{[Preparation]} In a new shell, go to the \texttt{simulation} folder. Run
the commands below in sequence to set up a new Minikube cluster and build our
application.

\begin{lstlisting}[language=bash]
# Create a new Minikube instance (1-5 minutes)
# Expected output:
# - A success message from Minikube
# - kubectl works correctly: try running `kubectl get nodes`
make run_minikube

# Build application from source (3 minutes)
# Expected output: No error messages
make build_minikube
\end{lstlisting}

\textit{[Execution]} To run all simulations at once, run the command below. Our
scripts will autonomously manage each simulation and collect all data.

\begin{lstlisting}[language=bash]
# Run all simulations (4.5-5 hours)
# Expected output:
# - a `simulations` folder created. `simulations/scenarios` contain one file for each run (16 in total)
# - (after 1 minute since start) the log file `simulations/out.log` does not show errors and is "SLEEPING"
# - (after 2-3 minutes since start) `kubectl -n v2x get pods` shows all pods in "Running" state
# - (after 4.5-5 hours since start) the log file `simulations/out.log` shows "ALL DONE" as last message
# - (after 4.5-5 hours since start) the `simulations/results` folder contains one file for each run (16 in total)
make run_simulations_background CONF=conf/ae.yaml
\end{lstlisting}

\textit{[Results]} Assuming that all simulations have been successful, the plots
can be generated by running \texttt{make plot\_all}. The results can be then
found in the \emph{simulations/figs} folder. Scenario A1 corresponds to
\cref{fig:eval-sim-base}, while the other scenarios are described in our Github
repository.

\subsubsection{Experiment (E3) - Claim (C3)}
\label{artifact:prl}
[Size of the \acs{PRL}] [10-20 human-minutes + 2.5-3.5 compute-hours]: The
experiment consists in creating a statistical model for the size of the
\ac{PRL}, where the process of adding and removing pseudonyms can be
approximated to a Markov model. Then, based on this model, we plot the expected
\ac{PRL} sizes in terms of percentiles of all possible states, using different
parameters such as the number of pseudonyms, the time each pseudonym needs to
stay in the \ac{PRL} (\paramtprl{}) and the share of attackers in the network.
See \cref{section:eval-prl-size} and Appendix~\ref{appendix:markov} for more
information.

\textit{[Preparation]} In a new shell, go to the \texttt{prl} folder.

\textit{[Execution]} All the data and plots can be computed with one single
command: \texttt{make all}. Alternatively, each plot can be computed separately
according to the instructions below.

\begin{lstlisting}[language=bash]
# probabilities and expected revocations (1-5 seconds)
# Expected output: results printed to standard output
make probabilities

# Fig. 15: transition graph (1-5 seconds)
# Expected output:
# - no errors printed to standard output
# - `tikz-graph.tex` plot under `./plots`
make tikz

# Fig 6: Series over different probabilities (15-20 min)
# Expected output: 
# - no errors printed to standard output
# - plots `p-plot_n800_e30.{tex,png}` under `./plots`
make p-plot

# Fig. 16: Series over different number of pseudonyms (25-35 min.)
# Expected output:
# - no errors printed to standard output
# - plots `n-plot_e30.{tex,png}` under `./plots`
make n-plot

# Fig. 17: Series over different T_prl (90-140 minutes)
# Expected output:
# - no errors printed to standard output
# - plots `t-plot_n800.{tex,png}` under `./plots`
make t-plot

# Fig. 7: Generate distribution for Tv (15-20 minutes)
# Expected output:
# - no errors printed to standard output
# - plots `tv-distribution.{tex,png}` under `./plots`
make tv-distribution
\end{lstlisting}

\textit{[Results]} Assuming that all commands have succeeded, all plots can be
found in the \emph{plots} folder.
