\section{Background}
\label{chapter:background}

% --------------------------------------------------------------------------
%\subsection{ITS and V2X Communications}
\label{background:v2x}

In the EU, performance and security requirements for \ac{ITS} and advanced use
cases (such as vehicle platooning or remote driving) are laid out
in~\cite{etsi20171331851400} and~\cite{etsi2020ts1221861620}. For example, it is
specified that \ac{ITS} need to support ``high connection density for congested
traffic'' with ``3,100 to 4,300 cars per square kilometer.'' Vehicles in these
scenarios communicate at rates of up to 50~messages/s with an end-to-end latency
of no more than 20~ms (platooning), or even via continuous data streams of up to
25 MBit/s with an end-to-end latency of no more than 5~ms (remote
driving)~\cite{etsi2020ts1221861620}.

To maintain security in these scenarios, infrastructure and vehicles must
implement efficient protocols that facilitate credential management, including
revocation, and signing and verifying message content while upholding privacy
objectives and legal requirements on data protection. As summarized by Hicks and
Garcia~\cite{hicks2020vehicular}, the requirements for the management of
\ac{V2X} credentials involve %the need for
communicating parties to verify the
authenticity of messages, guarantee unlinkability, minimize the repercussions of
a compromise of central trust management systems, and enable revocation of
vehicles upon the detection of policy violations, malfunction, or malicious
interaction. Therefore, vehicles must be capable to timely manage the validity
of credentials, which involves changing their own pseudonym certificates and
validating those of communicating parties. Importantly, communications and
security management protocols must also account for extended periods without
connectivity, where a vehicle is unreachable or powered off.

\noindent\textbf{\emph{Certificate Management and Revocation.}}
%
In Europe, the architecture for credential management in \ac{V2X} is defined by
ETSI~\cite{etsi2021102940,etsi2022102941}, while the US follows the \ac{SCMS}
design~\cite{brecht2018scms}. In both architectures, \ac{ITS} participants
possess two types of credentials: %
%
\begin{inparaenum}
    \item \emph{long-term credentials}, for authenticating a vehicle towards the
    \ac{ITS} infrastructure, and
    \item \emph{pseudonym credentials} (\emph{pseudonyms} in short), for secure
    and privacy-preserving peer-to-peer communication, such as \ac{V2V},
    \ac{V2I}, or \ac{V2P}.
\end{inparaenum} 
%
Long-term credentials can be either private keys stored in secure elements of
vehicles at manufacture time, or certificates obtained during enrollment with
the \ac{ITS}. Pseudonyms are typically retrieved from the infrastructure upon
successful authentication and authorization of a vehicle (via its long-term
credential), and are usually short-lived (with lifetime of up to
weeks~\cite{etsi2018tr04103415}) and rotated regularly to prevent vehicle
tracking.

Revocation aims at excluding a participant from interacting with others. We
distinguish between \emph{identity-based revocation} and \emph{message-based
revocation}, which differ in the way that the decision to revoke is made: The
former is based on a vehicle's \emph{canonical identity}, and is necessary to
enforce legal/policy requirements (e.g., to have only insured vehicles on the
road), in case of private-key compromise (e.g., a leak), or when a vulnerability
affecting a certain component or vehicle vendor becomes public, requiring
revocation of multiple vehicles at once. The latter, which is the focus of our
work, is based on peer-to-peer traffic, where a misbehaviour detection system is
in place to report faulty or malicious messages to a \ac{RA}, which eventually
mandates the revocation of the vehicle that signed such messages. When a vehicle
gets revoked, its long-term credentials are blacklisted by the \ac{ITS} in order
to deny it future requests, e.g., for re-enrollment or for obtaining new
pseudonyms. Additionally, all current vehicle's pseudonyms should be made
unusable, to prevent the transmission of malicious messages to other peers. In
this regard, \ac{SCMS} and ETSI use two opposing strategies, i.e., active and
passive revocation, respectively (cf.~\cref{related-work}).

\noindent\textbf{\emph{Trusted Execution and DAA.}}
%
A common approach to establish a root of trust in distributed systems is by
relying on \acfp{TC}, such as \acp{TPM} or \acp{TEE}~\cite{maene:hardware},
which provide the means to securely store and use long-term credentials, perform
key generation, remote attestation, and sealing.
%
Several papers propose the use of trusted components inside vehicles to manage
cryptographic credentials and increase security and
privacy~\cite{guette2009vanet,wagan2010vanet,jangid2022towards}.
%
Based on the security features provided by \acp{TC}, \acf{DAA} has been
developed as a privacy-preserving protocol to remotely authenticate a
system~\cite{brickell2004direct,tpm2015ieee11889}. In difference to ongoing
standardization efforts in \ac{ITS} and \ac{SCMS}, which rely heavily on
centralized public key
infrastructure~\cite{etsi2021102940,etsi2022102941,brecht2018scms}, the
application of \ac{DAA} in an \ac{ITS} context obsoletes most central PKI by
relying on secure processing elements (notably \acp{TPM}), which enable a
decentralized credential management. For example, some \ac{DAA} schemes allow
vehicles to derive pseudonyms autonomously from long-term credentials
\cite{whitefield2017privacy,desmoulins2019practical}. These pseudonyms are
self-certified using \ac{DAA} signatures that can be verified by other \ac{ITS}
participants. 

\section{Related Work}
\label{related-work}

Traditional message-based revocation schemes take two opposing directions:
\emph{Active revocation}, where the information of revoked pseudonyms is
distributed to vehicles in real time, and \emph{passive revocation}, where
pseudonyms have a short lifetime and are left to expire. Additionally, recent
schemes propose a novel approach based on \emph{self-revocation}.

\noindent\textbf{\emph{Active revocation.}}
%
Active revocation is employed by \ac{SCMS} and consists in the distribution of
\acp{CRL}~\cite{ietfcrl} to all vehicles in the network, containing the list of
pseudonyms that have been revoked and thus should not be trusted. Vehicles use
this information to decide whether or not to accept or discard a message
received from other peers. \ac{CRL} updates are issued regularly by the \ac{RA}
to indicate newly revoked pseudonyms, and must be received as quickly as
possible by all vehicles to ensure the safety of the system. However, as
\acp{CRL} grow large over time, the associated computation and communication
overheads make these mechanisms less suitable for real-time systems. Several
papers propose optimizations to reduce the size of \acp{CRL} and the
verification overhead~\cite{simplicio2019acpc,kumar2017binary}, though recent
surveys show that these optimizations are still far from
ideal~\cite{wang2020certificate,yoshizawa2022v2x_survey}. The \ac{SCMS} design
relies on \emph{linkage values} to identify all pseudonyms of a vehicle with a
single value, effectively reducing the size of the \ac{CRL} but significantly
increasing the delay to decide whether to accept or not a network
message~\cite{brecht2018scms}.

\noindent\textbf{\emph{Passive revocation.}}
%
As an alternative to active revocation, some schemes propose issuing pseudonym
certificates that have a fixed and short lifetime, forcing vehicles to request
new pseudonyms regularly to continue participating in the
network~\cite{hicks2020vehicular,verheul2019ifal,ahmed2017secure}. Internal
\acp{CRL} of long-term credentials are used by the infrastructure to deny such
requests to malicious vehicles, thus negating the need for pseudonym revocation.
%Here, a resolution mechanism is necessary to link pseudonym certificates to
%long-term identities of vehicles; However, this would mean that privacy is only
%guaranteed between vehicles, but not towards the infrastructure. 
Passive revocation is also adopted in the ETSI standard~\cite{etsi2022102941}.

Since pseudonym certificates are not revoked, vehicles detected as malicious can
continue their operation until the last of their pseudonyms has expired. As
such, malicious vehicles still have a non-negligible time window after their
detection in which they can spread false information and cause safety-related
incidents~\cite{etsi2018tr04103415}, and has been identified as an open
issue~\cite{yoshizawa2022v2x_survey}. Thus, to increase security, pseudonym
lifetimes would need to be quite short. However, with shorter lifetimes and
increasing participants in the network, the traffic between infrastructure and
vehicles would increase significantly, along with the risk of Sybil
attacks~\cite{douceur2002sybil}.
%
%In a technical report, ETSI carried out a study on pseudonym change management
%and rotation strategies based on time, number of signatures, and kilometers
%travelled \cite{etsi2018tr04103415}. In all cases, it appears that malicious
%vehicles still have a non-negligible time window after their detection in which
%they can spread false information and cause safety-related incidents.
%
%Existing work \cite{verheul2019ifal} tries to solve this problem by pre-issuing
%encrypted pseudonym certificates to vehicles. Such certificates can be
%decrypted and thus activated in the future by un-revoked vehicles via
%\emph{activation codes} received from the infrastructure. However, while this
%approach reduces the cost to generate and distribute certificates at runtime,
%it may increase the effective revocation time of malicious vehicles. For
%example, activation codes in IFAL \cite{verheul2019ifal} enable all
%certificates belonging to the same 90-day \emph{epoch}. 
%
%Recent work addresses this issue by combining active and passive
%revocation~\cite{simplicio2019acpc,kumar2017binary}, so that revoked pseudonyms
%only need to remain in the \ac{CRL} until their expiration time. However, this
%may still incur non-negligible network and verification overheads for high
%number of revocations.

\iffalse
Besides, pre-issuing certificates that are activated later in time requires
stronger assumptions in terms of how certificates are managed by a vehicle. For
example, IFAL relies on a fully trusted \ac{OBU} with access to a trusted time
source to enforce constraints on the usage of certificates, e.g., that each
certificate is only used for a pre-defined amount of time (e.g., 5 minutes),
then rotated and never used again.
\fi 

\noindent\textbf{\emph{Self-revocation.}}
%
Some \ac{DAA} protocols adopt a scheme called self-revocation, where revocation
is performed with the active cooperation of the vehicle to be
revoked~\cite{whitefield2017privacy,desmoulins2019practical,larsen2021daa}. The
concept of self-revocation was initially proposed by F\"orster et al. in their
\rewire{} protocol \cite{forster2015rewire}, as a privacy-preserving revocation
scheme for vehicular networks that leverage a hardware-based \ac{TC} in
vehicles.
%In this scheme, vehicles register to the
%Issuer when joining the network. After authentication of the \ac{TC},
%credentials are stored in secure memory, which can be used by the \ac{TC} to
%generate new pseudonyms independently, avoiding large communication overheads.
In such a scheme, the \ac{RA} broadcasts an \ac{OSR}, containing the pseudonym
identifier to be revoked. The idea is that, upon reception of such message, the
targeted vehicle will detect that the revoked pseudonym is one of its own and
then proceed with the deletion of all its credentials, such that the vehicle
will be unable to generate malicious messages any further. This approach
eliminates the need for \acp{CRL} or short pseudonym lifetimes, providing
substantial benefits in terms of performance and
scalability~\cite{angelogianni2023comparative}.

This simple scheme works with the assumption that the \ac{TC} is a trusted
entity and will behave as expected, i.e., that it will delete the credentials as
soon as the \ac{OSR} is received. In case such messages are delayed or dropped,
e.g., due to a network disruption or a malicious host, F\"orster et al.  propose
the use of \emph{keep-alive messages} that are periodically broadcast by the
\ac{RA} to vehicles and used to detect network interruptions, causing the
automatic deletion of all credentials in a \ac{TC} if no messages are received
for an extended period. However, this approach is described only informally, and
does not address an attack scenario where only \acp{OSR} are dropped, but
keep-alive messages are not. The \rewire{} scheme was formally verified in
\cite{whitefield2017formal}, though the possible disruption of \acp{OSR} was not
addressed. Furthermore, a similar \emph{heartbeat} mechanism is later proposed
in subsequent
work~\cite{whitefield2017privacy,desmoulins2019practical,larsen2021daa}, which
suffers from the same limitations. Finally, all solutions seem to implicitly
rely on a notion of trusted synchronized time to check the freshness of
revocation requests and compute timeouts, but such a feature may not be
available on all vehicles due to the limitations of current
technologies~\cite{alder2023about,anwar2019applications}, neither do previous
papers discuss possible issues related to clock drifting between vehicles and
the \ac{RA}.


%The use of
%secure processing elements further enables revocation schemes in which a
%vehicular \ac{TPM} is remotely instructed to purge or deprecate certain
%credentials, the execution of which can be remotely
%verified~(e.g.,~\cite{hicks2020vehicular,larsen2021daa}). This allows to make
%revocation an  efficient and decentralized process, similar to pseudonym
%generation, which avoids the management and distribution of \acp{CRL}. 
