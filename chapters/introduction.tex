\section{Introduction}

\acp{ITS} are a key technology for innovation in transportation that link
different modes of transportation and address safety, congestion, emissions,
and efficiency problems. Increased connectivity in vehicles, called \ac{V2X}
communication, incorporates digital interactions with other road users with
potentially safety-critical impacts on the vehicle and its passengers.
%  demand a secure and trustworthy infrastructure.
% In the EU, the ETSI standard provides several 
% to enable pseudonymity and
% unlinkability to protect privacy and personal data. 
% technical specification
% documents, such as~\cite{etsi2021102940,etsi2022102941,etsi2012ts102942} to 
% furthermore specify how certificates and pseudonymous certificates are 
% to be utilized in \ac{V2X} communication.
% can be used to achieve privacy goals. 
As highlighted by Sedar et al.~\cite{sedar2023v2x_survey} there are many
possible incentives for attackers to target \ac{V2X} systems, which range from
personal interest (e.g. tracking or insuarance fraud) to geopolitical
conflict (e.g. causing congestion or accidents).

A large body of academic research and technical standards for \ac{V2X}
communication focuses on privacy-preserving protocols to enroll and manage
pseudonymous
identities~\cite{whitefield2017privacy,desmoulins2019practical,larsen2021daa,etsi2021102940,etsi2022102941,brecht2018scms}.
% Besides, recent publications based on \ac{DAA} propose
% privacy-preserving \ac{V2X} schemes for road users, specifically vehicles, to
% enroll in \acp{ITS} and manage their pseudonymous
% identities~\cite{whitefield2017privacy,desmoulins2019practical,larsen2021daa}.
The key argument for such protocols is that special care must
be taken to ensure vehicle routes and patterns in their use of \ac{V2X}
services must not be observable by other
\ac{V2X} participants in the network.
Yet,
the central concern in transportation has always been road safety.
% which should 
%not be compromised by prioritizing privacy.
In \ac{V2X} scenarios, road safety requires that
misbehaving network participants, e.g., faulty or malicious vehicles
% that attempt to flood the network with messages to drown out benign communication, or 
that report inaccurate road and traffic conditions, can be identified and
prevented from causing harm. For example, the US \ac{SCMS} defines revocation
\emph{"an essential design objective"}~\cite{brecht2018scms}.

In a practical working system, it must be possible to 
\begin{inparaenum}
    \item detect, and
    \item punish 
\end{inparaenum}
such misbehaving participants by removing them from the network.
%
We identify a
common shortcoming in existing privacy-preserving \ac{V2X} approaches, with
respect to their capability to punish malicious or misbehaving actors after
misbehavior connected to a participant's pseudonym is detected:
Ensuring that this participant will not be able to communicate with other
vehicles in the future and effectively \emph{revoking} all their existing
credentials in a timely manner is either not guaranteed and avoidable by the
attacker, or is inefficient and not scalable due to bandwidth and computational
overheads.
% When vehicles are detected to be malfunctioning or malicious, harm to other 
% vehicles and passengers must be prevented by revoking any credentials the 
% malicious vehicle is communicating with.
% Depending on the utilized communication protocol, such credentials may be pseudonyms or
% any type of certificate or vehicle identities, and we will simply refer to any such 
% secret deployed to a vehicle used during communication as a credential.\fritz{Correct?}
% these vehicles
% must be excluded from interacting with \ac{ITS} infrastructure to maintain road
% safety. 
% Historically, the two common approaches to revoke credentials are \emph{active} and
% \emph{passive} revocation: The former involves centrally revoking the vehicles'
% pseudonymous identities and then propagating revocation information in the form
% of \acp{CRL} to \ac{ITS} participants~\cite{intelligent2016ieee16092}, while in
% the latter pseudonymous identities are left to expire and \acp{CRL} are used by
% the infrastructure to prevent revoked vehicles from re-enrolling with the
% \ac{ITS}~\cite{etsi2022102941}. As highlighted in recent
% surveys~\cite{yoshizawa2022v2x_survey,wang2020certificate}, revocation
% schemes for both approaches suffer from scalability issues, low
% reactiveness, or both.
%
Indeed, recent surveys~\cite{yoshizawa2022v2x_survey,wang2020certificate}
highlight that existing schemes based on active~\cite{brecht2018scms} or
passive~\cite{etsi2022102941} revocation suffer from scalability issues, low
reactiveness, or both.

To address these challenges, we build upon decentralized revocation schemes
based on
self-revocation~\cite{whitefield2017privacy,desmoulins2019practical,larsen2021daa},
which utilize a \ac{TC} embedded in vehicles for secure credential management.
Self-revocation enables vehicles to self-certify and also self-revoke
pseudonymous credentials, obsoleting most of the centralized public key
infrastructure of current \ac{ITS} designs. This paper addresses a number of
shortcomings of previous work, summarized by the following contributions:
%
\begin{enumerate}
%
    \item We design a secure self-revocation scheme that, in contrast to earlier
work, can guarantee revocation with an upper bound on revocation time, without
relying on a trusted time source in \acp{TC} (\Cref{chapter:design});
%
    \item We formalize and verify our design using the \tamarin{} prover
\cite{meier2013tamarin}, showing that it provides strong guarantees on the
revocation of credentials even in presence of a powerful adversary who drops
and/or delays revocation requests (\Cref{section:design-verification});
%
    \item We evaluate the security, performance, and scalability of our design,
showing that our approach guarantees prompt revocation and low utilization
of %computational and network
resources, even in the presence of network malfunctions and a high number of
attackers, in difference to related work
(\Cref{chapter:eval,chapter:discussion}).
%
    \item We discuss applicability of our revocation scheme to state-of-the-art
    \ac{V2X} protocols, showing that it is compatible with current standards in
    the EU and US, as well as \ac{DAA}
    (\Cref{section:discussion-compatibility}).
\end{enumerate}
%
% Beyond the use in \ac{ITS}, we believe that our scheme can be applied in
% scenarios where many devices make use of shared resources in a
% privacy-preserving but accountable way through pseudonymous identities, and
% where efficient and timely revocation of misbehaving or malicious entities is
% critical, e.g.,  in future 5G-edge scenarios or smart cities in general. 
%\fritz{I cut the 'beyond v2x' part from the intro...can we still fit it somehow?}
We provide the full formal specifications and an implementation of our design with this
publication~\cite{supplMaterial}. 
The most recent version of this artifact is also available on 
GitHub\footnote{\url{https://github.com/EricssonResearch/v2x-self-revocation}}.
