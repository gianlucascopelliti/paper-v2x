\section{Discussion}
\label{chapter:discussion}

Our design advances the state of the art by providing several advantages
compared to related work (\cref{related-work}), which we analyze in
\cref{section:discussion-comparison}. % Furthermore,
We discuss compatibility with
\ac{V2X} protocols such as ETSI, SCMS and \ac{DAA} in
\cref{section:discussion-compatibility}, and we explain a few corner cases of
our design in \cref{section:discussion-security}. Finally, we examine privacy
implications in \cref{section:discussion-privacy}.

\subsection{Qualitative Comparison with Related Work}
\label{section:discussion-comparison}

This section performs a qualitative analysis of network overhead, verification
overhead and revocation time of our approach compared with related work.

\noindent\textbf{\emph{Network overhead.}}
%
As discussed in \cref{section:eval-tv}, the bandwidth required to transmit
\acp{HB} is as low as a few kilobits per second in most cases, even considering
targeted attacks that cause a high number of revocations. This is a significant
improvement over active revocation schemes that leverage \acp{CRL}, since these
lists are ever growing and thus their size increases over time. We even
outperform schemes that combine active and passive revocation to reduce the
\ac{CRL} size: For example,~\cite{simplicio2019acpc} suggests periods in the
\ac{CRL} of 3-4 weeks for each pseudonym. Compared to our evaluation
(cf.~\cref{section:eval-tv}), this is still three to four orders of magnitude
higher.

Furthermore, unlike passive revocation schemes that require frequent pseudonym
changes~\cite{hicks2020vehicular}, our design consumes minimal resources at
runtime. Since \acp{HB} can be broadcast within an edge area, the number of
messages transmitted by the infrastructure for revocation only depends on the
number of edge areas and not on the number of vehicles. This makes the system
extremely scalable, as the \ac{RA} only needs to distribute relatively
small-sized \acp{HB} at low frequency (cf.~\cref{section:eval-tv}).
%%  Moreover, the \ac{RA} can generate \acp{HB} periodically and store
%% them in memory, so that pulling a \ac{HB} can be processed with extremely low
%% latency.

\noindent\textbf{\emph{Verification overhead.}}
%
Active revocation schemes incur some latency during the verification of \ac{V2V}
messages to check the revocation status of pseudonyms. Even though some
improvements exist to either reduce the \ac{CRL} size or optimize the
verification time, such schemes still cause a non-negligible overhead
(\cref{related-work}). Our design, instead, does not require any additional
verification overhead except for processing \acp{HB}, which are received more
seldomly than \ac{V2V} messages (cf.~\cref{section:eval-tv}). Thus, the overhead
for the \ac{TC} should be comparable to passive revocation schemes. % Although
In
\cref{section:design-extensions} we propose to use the \ac{PRL} to perform
active revocation, which is considered an \emph{optional} feature that can be
enabled only when the \ac{PRL} is small enough, which we demonstrated to be true
in most cases (\cref{section:eval-prl-size}).

\noindent\textbf{\emph{Revocation time.}}
%
Compared to active revocation, our design provides comparable results in the
average revocation time, especially when leveraging the extensions discussed in
\cref{section:design-extensions} and evaluated in \cref{section:eval-sim}.
However, active revocation does not have an upper bound on the effective
revocation time, as some vehicles may receive \ac{CRL} updates with
unpredictable delay. Besides, our scheme outperforms passive revocation since a
rotation of pseudonyms by the infrastructure every few minutes is infeasible
\cite{hicks2020vehicular}, while schemes relying on pre-issued certificates with
activation codes have a huge gap between revocation and actual removal of
malicious actors from the network \cite{verheul2019ifal,simplicio2019acpc}.

\subsection{Compatibility with Existing Protocols}
\label{section:discussion-compatibility}

\noindent\textbf{\emph{ETSI ITS.}}
%
In line with our system model~(cf.~\cref{section:system-model}), vehicles
following \emph{ETSI TS 102 941}~\cite{etsi2022102941} obtain long-term
\emph{enrollment credentials} from an Enrollment Authority, after successful
authentication of their canonical identity. These credentials are then used to
obtain pseudonymous \emph{authorization tickets} from an Authorization
Authority. Timestamps are used for replay protection~\cite{etsi2021102940}.
ETSI~\cite{etsi2022102941} further proposes passive revocation of pseudonyms by
centrally revoking long-term credentials~(cf.~\cref{chapter:background}). Our
scheme is applicable here and can largely improve on revocation time and network
overhead. This, however, requires \acp{TC} in all vehicles.

\noindent\textbf{\emph{SCMS.}}
%
Although similar to the ETSI architecture, the \ac{SCMS} scheme employs active
revocation (cf.~\cref{chapter:background}). Our scheme can entirely replace
\acp{CRL} in \ac{SCMS} to reduce the network utilization and nullify the
verification overhead that vehicles incur when processing \ac{V2V} messages,
which may be high due to the use of linkage values in the
\ac{CRL}~\cite{brecht2018scms}. Analogously to ETSI, the design of \ac{SCMS}
would need to include \acp{TC}.

\noindent\textbf{\emph{DAA.}}
%
On the academic side, \ac{DAA}-based protocols have been proposed as an approach
for pseudonym generation with strong privacy
properties~(cf.~\cref{chapter:background}). Vehicles use a \funcjoin{} protocol
to enroll, authenticating themselves via attestation to receive a long-term
credential. To obtain pseudonym certificates, some protocols require
communication with an Authorization Authority~\cite{hicks2020vehicular}; others
allow \acp{TC} to derive pseudonyms autonomously via an \funcissue{}
protocol~\cite{whitefield2017privacy,desmoulins2019practical}. While many
\ac{DAA} protocols do not explicitly discuss replay protection, we argue that
\cref{eq:valid-v2v-generic} is still a requirement of the \ac{V2X} network, and
thus provided. Under this assumption, even \ac{DAA} protocols are compatible
with our revocation scheme.

\ac{DAA} protocols that already propose self-revocation can benefit from our
design by having stronger guarantees on revocation time. However, protocols
where \acp{TC} derive pseudonyms
directly~\cite{whitefield2017privacy,desmoulins2019practical} might not satisfy
\cref{req:revoked} because the infrastructure cannot map pseudonyms to long-term
identities, and therefore revoked vehicles might be able to re-enroll the
network. While this has clear benefits in terms of privacy, it raises security
concerns (cf.~\cref{section:discussion-privacy}).

\noindent\textbf{\emph{Applicability beyond \ac{V2X}.}}
%
Our revocation scheme can be generalized to revoke \emph{any} type of
credential. Hence, it may be applied to other use cases, e.g., in scenarios
where
\begin{inparaenum}
\item real-time communication requirements hamper the use of classic revocation
  checks,
\item users may generate pseudonymous credentials for enhanced privacy
  w.r.t.~peers or the infrastructure, or
\item revocation needs to be fast and reliable to avoid serious damage by bad
  actors.
\end{inparaenum}
V2X networks check all boxes but we see benefits for, e.g., smart
cities, direct mobile-to-mobile communication, and other peer-to-peer
applications.

\subsection{Security Considerations}
\label{section:discussion-security}

\noindent\textbf{\emph{Vehicles powered off.}}
%
Vehicles that are temporarily powered off cannot process any \acp{HB}, meaning
that potential revocations may be missed without the \ac{TC} noticing, and
without being able to execute automatic revocation. Thus, it is important to
persist the value of the last \paramthb{} processed by a \ac{TC}, such that
vehicles would remain synchronized when powered off for a short time. For longer
offline periods, instead, vehicles would need to re-synchronize with the
infrastructure and, simultaneously, check their revocation status.

\noindent\textbf{\emph{Self-identification of pseudonyms.}}
%
The underlying assumption of our revocation scheme is that,  to perform
self-revocation, a \ac{TC} is able to identify its own pseudonym certificates
within the \ac{PRL}. However, it may not be always possible to do so when the
\ac{TC} occasionally deletes some pseudonyms from memory, e.g., due to pseudonym
rotation. A trivial solution to this problem would be to never delete any
pseudonyms, though this may not be possible in resource-constrained environments
(e.g., \acp{TPM}). Previous work~\cite{forster2015rewire,whitefield2017formal}
discusses a more efficient solution that leverages cryptographic tokens,
allowing \acp{TC} to safely delete their pseudonyms.

\noindent\textbf{\emph{Denial-of-Service Attacks.}}
%
An attacker may cause vehicles to get de-synchronized via network attacks,
preventing them from processing \acp{HB}. This could, e.g., be achieved by
jamming all wireless network activity close to a victim. While not being
practical at scale, jamming attacks would likely disconnect the victim from the
network regardless of the credential management mechanisms employed, and may
thus require re-enrollment or other connection management steps. Therefore, such
attacks likely do not cause additional harm when our approach is used.

\subsection{Privacy Discussion}
\label{section:discussion-privacy}

Privacy requirements in V2X include that adversaries cannot link a pseudonymous
signature to a canonical identity or other pseudonyms, or observe the use of
specific network resources~\cite{whitefield2017privacy, etsi2022102941,
yoshizawa2022v2x_survey}. Aside from adversarial interactions, data minimization
towards the infrastructure is also a design goal of V2X systems to minimize the
overall attack surface~\cite{brecht2018scms}.

Our design does not compromise the privacy provisions of the underlying
certificate management steps -- enrollment, pseudonym generation, etc. --
because \acp{HB} only contain public information and an adversary receiving
\acp{PRL} only sees a list of pseudonyms (similarly to traditional \acp{CRL})
but learns no further information about specific vehicles, nor can they link
pseudonyms to each other. Learning when specific pseudonyms are considered
compromised by the infrastructure is unavoidable, and breaking the underlying
cryptographic primitives is outside of our attacker model.

However, \cref{req:revoked} and technical standards in
Europe~\cite{etsi2022102941} and the US~\cite{brecht2018scms} highlight that
revoked vehicles must no be able to re-enroll. This compromises data
minimization in the V2X infrastructure since it necessitates mapping a pseudonym
back to the canonical identity or long-term credentials of the vehicle and then
deny-listing that identity.
%
There are DAA-based approaches that preserve unlinkability of pseudonyms towards
the infrastructure, as \acp{TC} generate pseudonyms
autonomously~\cite{whitefield2017privacy,desmoulins2019practical}. However, this
in turn raises security concerns since re-enrollment of revoked vehicles becomes
feasible. Possible technical solutions to this conundrum depend on the V2X
protocol and its credential management scheme, as well as the capabilities of
the TC. Therefore they are out of scope of this work.
